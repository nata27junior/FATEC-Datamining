    \newpage
\chapter{Conclusão}
\par Este trabalho foi desenvolvido para atender as necessidades da empresa TecSUS, utilizando os dados da empresa para realizar um estudo no consumo de água para investigação de detecção possível excesso e vazamento em residências e empresas, visando economia no consumo de água. Utilizando  algumas técnicas de \emph{Data Mining} para plotagem de gráficos, e aplicando a fórmula da Distância Euclidiana. 

\par Tendo em vista  os experimentos aplicados nos dados e os resultados dos gráficos de comparação nos dias da semana, afirma-se que  em alguns períodos não há consumo de água, sendo assim nesses períodos  não houve vazamentos.
 
\par Reforçando que na média geral de todo o arquivo o consumo ficou menor que 30 litros, e nas médias diárias pode se observar que o consumo fica abaixo de 50 litros. E que no acumulado diário o consumo ultrapassa os 800 litros.

\par E aplicando o algoritmo de plotagem nos dados de simulação de vazamento figura \ref{graf_simula_vaz}, obtém um gráfico de simulação de um vazamento, notando que houve um aumento expressivo no consumo de água e retornando ao estado de normalidade quando o problema estiver sanado. E observando  na figura \ref{simula_vaz} onde é feita a soma do total dos litros consumidos e dos litros da simulação do vazamento, e veja  a diferença entre os dois resultados, é verificado que tem um desperdício e assim ocasionando um aumento relativo na conta. E na figura \ref{graf_simula_vaz_d} onde é aplicada a Distância Euclidiana observa se que houve um deslocamento na linha apontando o  aumento no consumo de água ocasionado pelo vazamento, e retornando a normalidade depois que o problema foi sanado.

\par E ao utilizar a fórmula matemática da Distância Euclidiana, que é muito utilizada para análise de séries temporal, o resultado tende ser o mais próximo de zero. E que quanto maior o valor observado menos semelhante será a séries temporal. E que foi possível observar os resultados no decorrer desses dados, que há períodos semelhantes no consumo.

\par E  para realizar previsões para advertir o consumidor caso tivesse um excesso no consumo é necessário uma análise mais profunda, com mais período de dados.


\section{Trabalhos Futuros}
\par Este trabalho expande a possibilidade de melhorias e oportunidade para novas implementações das técnicas. Sendo eles:
\begin{enumerate}
\item  Estudar mais a fundo os dados de  consumo para poder obter mais detalhes para que se possa obter talvez uma previsão nos dados.
\item  Aplicar as técnicas de data mining para outro serviço da empresa TecSUS como sendo o TecLux;
\item  Aplicar as técnicas de data mining para outro serviço da empresa TecSUS como sendo o TecGas;  
\end{enumerate}
   
   
    
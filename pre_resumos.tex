% resumo em português
\setlength{\absparsep}{18pt} % ajusta o espaçamento dos parágrafos do resumo
% --- resumo em português ---
\begin{resumo}
O Estado de São Paulo sofreu em 2014 com problemas da crise hídrica e assim despertou-se a necessidade de armazenamento e o consumo inteligente de água.
A empresa TecSUS, visando à sustentabilidade, desenvolve tecnologias que ajudam na economia dos recursos. Entre seus produtos se destaca TecHydro, equipamento que faz a captação dos dados de consumo de água. A empresa TecSUS com os dados obtidos, necessitava de uma análise para uma possível detecção de vazamento e  excesso no consumo de água, visando economia de água. Assim inicia-se um estudo com os dados utilizando algumas técnicas de \emph{Data Mining}, ou melhor, Mineração de Dados, com ferramentas e bibliotecas \emph{open source}, com a linguagem de programação \emph{Python} e com utilização do \emph{Jupyter Notebook} que é uma ferramenta \emph{Web} que auxilia na integração dos algoritmos, das bibliotecas e facilita a visualização dos resultados.
E com aplicação da fórmula matemática da Distância Euclidiana, verificou que houve períodos sem consumo de água e sendo assim podendo constatar de que nesses períodos não houve vazamento de água.
    
    \vspace{\onelineskip}
    \noindent
    \textbf{Palavras-chave}: Data Mining; Distância Euclidiana; TecHydro; \emph{Jupyter Notebook}; \emph{Python}.
\end{resumo}

% resumo em inglês
\begin{resumo}[Abstract]
    \begin{otherlanguage*}{english}

\textit{
The State of São Paulo suffered in 2014 with problems of the water crisis and thus the need for storage and the intelligent consumption of water was awakened.
 The TecSUS company, aiming at sustainability, develops technologies that help in saving resources. Among its products stands out TecHydro, equipment that captures water consumption data. The company TecSUS with the obtained data, needed an analysis for a possible detection of leakage and excess in the consumption of water, aiming at saving water. Thus, a study of the data using some techniques of Data Mining, or better, Data Mining, with tools and open source libraries, with the programming language Python and with using the Jupyter Notebook which is a Web tool that helps in the integration of algorithms, libraries and facilitates the visualization of the results.
 And with the application of the mathematical formula of Euclidean Distance, it was verified that there were periods without water consumption and being thus able to verify that in those periods there was no water leakage.
}%

	    \vspace{\onelineskip}
	    \noindent
	    \\
	    \textbf{Keywords}: Data Mining; Distância Euclidiana; TecHydro; Jupyter Notebook; Python.
    \end{otherlanguage*}
\end{resumo}